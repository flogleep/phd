\documentclass[c]{beamer}

\usepackage[utf8]{inputenc}
\usepackage[french]{babel}

\usepackage{amsfonts}
\usepackage{amsmath}

\usetheme{Warsaw}

\title{Blah}
\author{Igor Colin}
\date{\today}

\setbeamertemplate{navigation symbols}{}

\AtBeginSection[]
{
    \begin{frame}<beamer>
        \tableofcontents[currentsection]
    \end{frame}
}

\begin{document}

\maketitle

\section{Données à disposition}
\begin{frame}
    \begin{itemize}
        \item Données de départs : graphe utilisateurs
        \item N\oe{}uds : utilisateurs
        \item Arrêtes : relations d'amitié, influence, fréquence de partage

    \end{itemize}
\end{frame}

\section{Génération dynamique de graphe}
\begin{frame}
    \begin{itemize}
        \item Graphe d'utilisateurs dynamique
        \begin{itemize}
            \item Nouveaux utilisateurs, départ d'utilisateurs
            \item Nouvelles connexions entre utilisateurs, disparition de
                connexions
            \item Évolution des connexions existantes (influence dynamique)
        \end{itemize}
    \end{itemize}
\end{frame}

\begin{frame}
    \begin{itemize}
        \item Objectif : prédire l'évolution du graphe
        \item Définir un modèle d'évolution dynamique
        \item Utiliser l'historique des évolutions pour retrouver
            les paramètres du modèle
        \item Mettre à jour ces paramètres en ligne ou à intervalle fixé
    \end{itemize}
\end{frame}

\section{Renforcement}
\begin{frame}
    \begin{itemize}
        \item Possibilité d'interagir avec l'utilisateur via des suggestions
        \item Comment maximiser l'impact de ces suggestions ?
        \item Processus de décision markovien :
        \begin{itemize}
            \item Ensemble d'états (connexions du graphe, satisfaction
                utilisateur)
            \item Ensemble de décisions possibles (suggestions)
            \item Déduction d'un politique optimale
        \end{itemize}
    \end{itemize}
\end{frame}

\end{document}
