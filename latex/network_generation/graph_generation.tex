%        File: graph_generation.tex
%     Created: Tue Feb 11 11:00 AM 2014 C
% Last Change: Tue Feb 11 11:00 AM 2014 C
%
\documentclass[a4paper]{article}

\usepackage[utf8]{inputenc}
\usepackage[french]{babel}

\usepackage[]{amsfonts}
\usepackage[]{amsmath}

\usepackage{algpseudocode}
\usepackage{algorithm}

\title{Génération de graphes}
\author{Igor Colin}
\date{\today}

\begin{document}

\section{Méthodes MCMC}
\label{sec:methodes_mcmc}
    Soit $G = (V, E)$ un graphe non orienté. Pour $i \in V$, on note $d_i$ le
    degré du n\oe{}ud $i$. On note également $D^{(1)}$ la distribution des
    degrés du graphe. Autrement dit, pour $n \in \mathbb{N}$,
    \[
        D^{(1)}(n) = \frac{|\{i \in V, d_i = n\}|}{|V|}.
    \]
    Enfin, on note $D^{(2)}$ la distribution jointe des degrés, définie pour
    $(n, p) \in \mathbb{N} \times \mathbb{N}$ par :
    \[
        D^{(2)}(n, p) = \frac{|\left\{ (i, j) \in E, (d_i, d_j) = (n, p)
            \text{ ou } (d_i, d_j) = (p, n)\right\}|}{|E|}.
    \]
    L'objectif des méthodes MCMC pour la génération de graphe est de partir
    d'un graphe existant et de modifier aléatoirement ses arrêtes, tout en
    conservant la distribution jointe des degrés initiale. Pour ce faire,
    cette méthode procède comme indiqué dans l'Algorithme~\ref{alg:mcmc}.

    \begin{algorithm}
        \caption{Algorithme MCMC pour la génération de graphe.}
        \label{alg:mcmc}
        \begin{algorithmic}
            \Require $V, E, N_{max}$
            \State $E^{(0)} \gets E$
            \State $i \gets 1$
            \While{$i \leq N_{max}$}
            \State $(e_1, e_2) \gets \text{rand}(E^{(i-1)})$
            \State $(f_1, f_2) \gets \text{rand}(\left\{ (u, v) \in E^{(i-1)}, d_u = d_{e_1} \right\})$
            \State $E^{(i)} \gets E^{(i-1)} \backslash \left\{ (e_1, e_2), (f_1, f_2) \right\}$
            \State $E^{(i)} \gets E^{(i)} \cup \left\{ (e_1, f_2), (f_1, e_2) \right\}$
            \State $i \gets i + 1$
            \EndWhile \\
            \Return $(V, E^{(N_{max})})
        \end{algorithmic}
    \end{algorithm}

\end{document}


