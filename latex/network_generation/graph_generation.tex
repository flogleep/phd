%        File: graph_generation.tex
%     Created: Tue Feb 11 11:00 AM 2014 C
% Last Change: Tue Feb 11 11:00 AM 2014 C
%
\documentclass[a4paper]{article}

\usepackage[utf8]{inputenc}
\usepackage[french]{babel}

\usepackage[]{amsfonts}
\usepackage[]{amsmath}

\usepackage{algpseudocode}
\usepackage{algorithm}

\title{Génération de graphes}
\author{Igor Colin}
\date{\today}

\begin{document}

\section{Graphes aléatoires}
\label{sec:graphes_aleatoires}

    \subsection{Loi de Poisson}
    \label{sub:loi_de_poisson}
        Cette catégorie de modèle effectue très peu d'hypothèses sur les
        graphes à générer et permet d'atteindre une grande quantité de graphes
        d'une taille donnée. Ces modèles possèdent deux paramètres. Le premier
        paramètre indique le nombre de n\oe{}uds du graphe à générer. Le
        deuxième paramètre porte sur les arrêtes du graphe ; il n'indique pas
        nécessairement le nombre d'arrêtes générées mais indique au moins leur
        probabilité d'occurence.

        \subsubsection{Modèle de Gilbert}
            Pour $n \in \mathbb{N}^*$ et $0 < p < 1$, le modèle de Gilbert,
            génère un graphe composé de $n$ n\oe{}uds dont chaque arrête
            potentielle a une probabilité $p$ d'être présente, indépendamment
            des autres arrêtes.
            En général, on choisit $p$ en fonction de $n$ et typiquement
            $p(n) = o(1)$. \`A noter que dans cette méthode, on ne peut
            connaître à l'avance le nombre d'arrêtes du graphe généré.
            Gr\^ace à l'analyse de ce type fournie par \cite{newman2001random},
            on sait que la probabilité $p_k$ qu'un n\oe{}ud pris au hasard
            soit de degré $k$ vaut :
            \[
                p_k = \binom{|V|}{k} p^k (1 - p)^{N - k}.
            \]
            En notant $z = \mathbb{E}\left[|E|\right] = |V|p$, on a le résultat
            suivant :
            \[
                p_k \underset{N \rightarrow +\infty}{\sim}
                \frac{z^k e^{-z}}{k!},
            \]
            ce qui indique que ce modèle tend asymptotiquement vers une
            distribution des degrés suivant une loi de Poisson.

        \subsubsection{Modèle d'Erdös-Rényi}

\section{Méthodes MCMC}
\label{sec:methodes_mcmc}
    Soit $G = (V, E)$ un graphe non orienté. Pour $i \in V$, on note $d_i$ le
    degré du n\oe{}ud $i$. On note également $D^{(1)}$ la distribution des
    degrés du graphe. Autrement dit, pour $n \in \mathbb{N}$,
    \[
        D^{(1)}(n) = \frac{|\{i \in V, d_i = n\}|}{|V|}.
    \]
    Enfin, on note $D^{(2)}$ la distribution jointe des degrés, définie pour
    $(n, p) \in \mathbb{N} \times \mathbb{N}$ par :
    \[
        D^{(2)}(n, p) = \frac{|\left\{ (i, j) \in E, (d_i, d_j) = (n, p)
            \text{ ou } (d_i, d_j) = (p, n)\right\}|}{|E|}.
    \]
    Les méthodes MCMC pour la génération de graphes sont très vastement
    utilisées (\cite{rao1996markov}). Leur principe général est de partir
    d'un graphe existant et de modifier aléatoirement ses arrêtes, tout en
    conservant certaines propriétés du graphe. L'Algorithme~\ref{alg:mcmc_deg}
    sélectionne aléatoirement deux arrêtes, puis les échange si cela
    n'entraine ni la création de boucle ni de double arrête.

    \begin{algorithm}
        \caption{Algorithme MCMC conservant les degrés des n\oe{}uds.}
        \label{alg:mcmc_deg}
        \begin{algorithmic}[1]
            \Require $V, E, N_{max}$
            \State $E^{(0)} \gets E$
            \State $i \gets 1$
            \While{$i \leq N_{max}$}
                \State $(e_1, e_2) \gets \text{rand}(E^{(i-1)})$
                \State $(f_1, f_2) \gets \text{rand}(E^{(i-1)})$
                \If{no loop and no double edge created}
                    \State $E^{(i)} \gets E^{(i-1)} \backslash \left\{ (e_1, e_2), (f_1, f_2) \right\}$
                    \State $E^{(i)} \gets E^{(i)} \cup \left\{ (e_1, f_2), (f_1, e_2) \right\}$
                \Else
                    \State $E^{(i)} \gets E^{(i-1)}$
                \EndIf
                \State $i \gets i + 1$
            \EndWhile
        \end{algorithmic}
    \end{algorithm}

    Cet algorithme est simple à implémenter et ne nécessite qu'un paramètre :
    le nombre d'itérations $N_{max}$. En général, on choisit $N_{max}$ de
    l'ordre de $100 \times |E|$ pour s'assurer un graphe généré indépendant de
    l'original. Cependant, ce choix est purement expérimental et il existe peu
    de travaux théoriques sur ces critères ; \cite{ray2012we} présente
    toutefois des résultats intéressants sur des critères d'arrêts spécifiques.

    \begin{algorithm}
        \caption{Algorithme MCMC pour la génération de graphe.}
        \label{alg:mcmc_joint}
        \begin{algorithmic}
            \Require $V, E, N_{max}$
            \State $E^{(0)} \gets E$
            \State $i \gets 1$
            \While{$i \leq N_{max}$}
            \State $(e_1, e_2) \gets \text{rand}(E^{(i-1)})$
            \State $(f_1, f_2) \gets \text{rand}(\left\{ (u, v) \in E^{(i-1)}, d_u = d_{e_1} \right\})$
            \State $E^{(i)} \gets E^{(i-1)} \backslash \left\{ (e_1, e_2), (f_1, f_2) \right\}$
            \State $E^{(i)} \gets E^{(i)} \cup \left\{ (e_1, f_2), (f_1, e_2) \right\}$
            \State $i \gets i + 1$
            \EndWhile \\
            \Return $(V, E^{(N_{max})})$
        \end{algorithmic}
    \end{algorithm}

    \bibliographystyle{alpha}
    \bibliography{network}

\end{document}


