%        File: problem_formulation.tex
%     Created: Tue Apr 15 12:00 PM 2014 C
% Last Change: Tue Apr 15 12:00 PM 2014 C
%
\documentclass[a4paper]{article}

\usepackage[utf8]{inputenc}
\usepackage[french]{babel}

\usepackage{amsmath}
\usepackage{amsfonts}

\DeclareMathOperator*{\argmin}{arg\,min}
\DeclareMathOperator*{\argmax}{arg\,max}

\begin{document}

Soit $G = (V, E)$ un graphe d'utilisateurs. En notant $N$ le cardinal de $V$, on
identifie ce dernier avec $\{1,\ldots,N\}$. On considère dans un premier temps
des arrêtes sans poids, soit $E \subset V \times V$.

Chaque utilisateur $v \in V$ est caractérisé par un vecteur
$\mathbf{X}_v \in \mathbb{R}^d$ pour un certain $d > 0$. On suppose pour l'instant
que la dimension $d$ est la même pour tous les utilisateurs.

Pour résoudre les différents problèmes posés, nous avons à disposition
$N + 1$ agents : les $N$ sommets du graphe et un serveur extérieur, que l'on
indicera par $0$. Pour représenter l'information disponible pour chaque agent,
on décompose, pour un utilisateur $v \in V$ donné, le vecteur $\mathbf{X}_v$ sous
la forme suivante :
\[
    \mathbf{X}_v = \left[ \mathbf{X}_v^{(1)}, \mathbf{X}_v^{(2)}, \mathbf{X}_v^{(3)} \right],
\]
avec $\mathbf{X}_v^{(j)} \in \mathbb{R}^{d_j}$ pour $1 \leq j \leq 3$ et $\sum_{j=1}^3 d_j = d$.
La première partie représente l'information accessible par les utilisateurs
connectés à $v$. La seconde partie représente l'information supplémentaire
accessible par le serveur. Enfin, la dernière partie représente les informations
n'étant accessibles qu'à $v$ lui-même.

L'agent $0$ a à sa disposition les informations suivantes :
\begin{itemize}
    \item $\{\mathbf{X}_i^{(1)}, \; i \in V\}$
    \item $\left\{\mathbf{X}_i^{(2)}, \; i \in V \right\}$
\end{itemize}
Un agent $i \in \{1,\ldots,N\}$ a quant à lui accès aux informations suivantes :
\begin{itemize}
    \item $\mathbf{X}_i$
    \item $\left\{ \mathbf{X}_j^{(1)}, \; (i, j) \in E \right\}$
\end{itemize}

L'objectif est d'identifier des communautés dans le graphe, en utilisant les
données accessibles à chaque n\oe{}ud du graphe, tout en limitant l'échange
d'information entre les différents n\oe{}uds.

Pour $C \subset V$, on définit la fonction $\mathbf{1}_C : V \rightarrow \{0, 1\}$
qui à tout élément $v$ de $V$ associe $1$ si $v \in C$ et $0$ sinon.

\section{Utiliser les attributs des n\oe{}uds}
\label{sec:utiliser_attributs_noeuds}
Une première approche naïve pour arriver à construire des communautés
pouvant être \og{} étiquetées \fg est de se concentrer sur les attributs des différents
n\oe{}uds : on souhaiterait regrouper les utilisateurs ayant des intérêts communs
sans forcément se préoccuper de leur proximité dans le graphe.

\subsection{Clustering}
\label{sub:clustering}
Le \emph{clustering} consiste à trouver une partition $\mathcal{P}$ de $V$ minimisant
une inertie $W(\mathcal{P})$. On définit par cette intertie par :
\[
    W(\mathcal{P}) = \frac{2}{n(n-1)} \sum_{1 \leq i < j \leq N} D(\mathbf{X}_i, \mathbf{X}_j) \Phi_{\mathcal{P}}(i, j),
\]
où $D(\mathbf{X}_i, \mathbf{X}_j)$ est une mesure de dissimilarité entre
$\mathbf{X}_i$ et $\mathbf{X}_j$ et $\Phi_{\mathcal{P}}(i, j)$
vaut $1$ si $i$ et $j$ sont dans le même \emph{cluster}, $0$ sinon.

Dans notre cas, si l'on ne s'intéresse pas aux connexions entre les utilisateurs,
on peut par exemple se baser sur la similarité cosinus pour définir :
\[
    D:\left\{
        \begin{array}{r c l}
            V \times V & \rightarrow & [0, 1] \\
            (X_i, X_j) & \mapsto & \frac{1}{2} - \frac{\mathbf{X}_i^T \mathbf{X}_j}{2\|\mathbf{X}_i\|_2 \|\mathbf{X}_j\|_2}
        \end{array}
    \right.
\]
Cette définition pose cependant un problème. En effet, il n'est possible pour aucun
agent de calculer explicitement $D(X_i,X_j)$, pour tout $i \neq j$, car aucun agent n'a accès
simultanément à $\mathbf{X}^{(3)}_i$ et $\mathbf{X}^{(3)}_j$. Soit $D^{(1)}$ l'approximation \og{} au
premier ordre \fg de D :
\[
    D^{(1)}:\left\{
        \begin{array}{r c l}
            V \times V & \rightarrow & [0, 1] \\
            (X_i, X_j) & \mapsto & \frac{1 - {\mathbf{X}^{(1)}_i}^T \mathbf{X}^{(1)}_j}{2}
        \end{array}
    \right.
\]
Désormais, si $(i, j) \in E$, alors les agents $i$ et $j$ peuvent calculer
$D^{(1)}(\mathbf{X}_i, \mathbf{X}_j)$.

\end{document}


