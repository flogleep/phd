\documentclass[c]{beamer}

\usepackage[utf8]{inputenc}
\usepackage[french]{babel}

\usepackage{amsfonts}
\usepackage{amsmath}

\newtheorem*{deffr}{Définition}
\newtheorem*{propriete}{Propriété}
\newtheorem*{theofr}{Théorème}

\usepackage{graphicx}
\usepackage{array}

\usetheme{Warsaw}

\title{Optimisation distribu\'ee}
\author{Igor Colin}
\date{\today}

\setbeamertemplate{navigation symbols}{}

\AtBeginSection[]
{
    \begin{frame}<beamer>
        \tableofcontents[currentsection]
    \end{frame}
}

\begin{document}

\begin{frame}
    \begin{description}
        \item[Contexte]
            \begin{itemize}
                \item Ensemble d'utilisateurs
                \item Caract\'eristiques
                \item Calcul potentiel
            \end{itemize}
        \item[Objectifs]
            \begin{itemize}
                \item Regrouper les utilisateurs par caract\'eristiques communes
                \item R\'epartir le calcul
            \end{itemize}
        \item[Contraintes]
            \begin{itemize}
                \item Informations priv\'ees
                \item Limitation de la communication
            \end{itemize}
    \end{description}
\end{frame}

\section{Mod\'elisation}
\begin{frame}
    \frametitle{Notations}
    \begin{description}
        \item[R\'eseau des utilisateurs] :
            
            $G = (V, E)$, $V = \{1, \ldots, N\}$
        \item[Caract\'eristiques] :
            
            $\left( X_v \right)_{1 \leq v \leq N}$
    \end{description}
\end{frame}

\begin{frame}
    \frametitle{Formulation}
    Problème à r\'esoudre :

    \[
        \begin{aligned}
            & \min_{\mathcal{P}}
            && W\left(\mathcal{P} \right)
            = \frac{2}{n(n-1)} \sum_{1 \leq i < j \leq N} D(X_i, X_j) \Phi_{\mathcal{P}}(X_i, X_j),
        \end{aligned}
    \]

    \begin{description}
        \item[Dissimilarit\'e des caract\'eristiques] :

            \begin{itemize}
                \item
                    $D:\left\{
                    \begin{array}{r c l}
                        V \times V & \rightarrow & \mathbb{R}_+ \\
                        (x, y) & \mapsto & D(x, y)
                    \end{array}
                    \right.$
                \item $D(x, y) = 0 \iff x = y$
                \item $D(x, y) = \|x - y\|_p$
            \end{itemize}
        \item[Proximit\'e dans la partition] :
            $\mathcal{P} = \left( C_1, \ldots, C_k \right)$

            $\Phi_{\mathcal{P}}(i, j) = \mathbf{1}_{\{i \text{ et } j \text{ sont dans le même
            cluster}\}}$

    \end{description}
\end{frame}

\end{document}
