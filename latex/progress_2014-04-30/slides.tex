\documentclass[c]{beamer}

\usepackage[utf8]{inputenc}
\usepackage[french]{babel}

\usepackage{amsfonts}
\usepackage{amsmath}

\newtheorem*{deffr}{Définition}
\newtheorem*{propriete}{Propriété}
\newtheorem*{theofr}{Théorème}

\usepackage{graphicx}
\usepackage{array}

\usetheme{Warsaw}

\title{Optimisation distribu\'ee}
\author{Igor Colin}
\date{\today}

\setbeamertemplate{navigation symbols}{}

\AtBeginSection[]
{
    \begin{frame}<beamer>
        \tableofcontents[currentsection]
    \end{frame}
}

\begin{document}

\begin{frame}
    \begin{description}
        \item[Contexte]
            \begin{itemize}
                \item Ensemble d'utilisateurs
                \item Caract\'eristiques utilisateurs observ\'ees
                \item Potentiel de calcul des utilisateurs
            \end{itemize}
        \vspace{.3cm}
        \item[Objectifs]
            \begin{itemize}
                \item Regrouper les utilisateurs en tribus
                \item R\'epartir le calcul
            \end{itemize}
        \vspace{.3cm}
        \item[Contraintes]
            \begin{itemize}
                \item Informations priv\'ees
                \item Limitation de la communication
            \end{itemize}
        \vspace{.3cm}
    \end{description}
\end{frame}

\section{Mod\'elisation}
\begin{frame}
    \frametitle{Notations}
    \begin{itemize}
        \item R\'eseau des utilisateurs $G = (V, E), \; V = \{1, \ldots, N\}$
            \vspace{.3cm}
        \item Caract\'eristiques $\mathbf{X} = \left( X_i \right)_{1 \leq i \leq N}$
            \vspace{0.15cm}

            ex: SMS, a écouté [XXX], a partagé [YYY]
            \vspace{.3cm}
        \item Distance entre utilisateurs :
            \[
                        D:\left\{
                        \begin{array}{r c l}
                            V \times V & \rightarrow & \mathbb{R}_+ \\
                            (x, y) & \mapsto & D(x, y)
                        \end{array}
                        \right.
                    \]

                    $D(x, y) = 0 \iff x = y$
                
                    $D(x, y) = \|x - y\|_p$


    \end{itemize}
\end{frame}

\begin{frame}
    \frametitle{Formulation}
    \begin{itemize}
        \item Objectif : partitionner en tribus, par affinités
        \item
            Problème à r\'esoudre : trouver une partition $\mathcal{P}^*$ solution de

    \[
        \begin{aligned}
            & \min_{\mathcal{P}}
            && W\left(\mathcal{P} \right)
            = \frac{2}{N(N-1)} \sum_{1 \leq i < j \leq N} D(X_i, X_j) \Phi_{\mathcal{P}}(i, j),
        \end{aligned}
    \]


    $\Phi_{\mathcal{P}}(i, j) = \left\{
        \begin{array}{l l}
            1& \text{ si $i$ et $j$ dans le même cluster} \\
            0 & \text{ sinon}
        \end{array}
            \right.
            $

    \end{itemize}

\end{frame}

\section{Résolution}
\begin{frame}
    \frametitle{Contraintes}

    \begin{itemize}
        \item Contraintes supplémentaires :
        \begin{itemize}
            \item Distribution du calcul
            \item Limitation du transfert de données
        \end{itemize}
    
        \item Problème d'optimisation distribuée
    \end{itemize}

\end{frame}

\begin{frame}
    \frametitle{Énoncé}

    Nouvelle formulation du problème :
    \[
        \begin{aligned}
            & \min_{\mathcal{P}}
            && W\left(\mathcal{P} \right)
            = \frac{1}{N} \sum_{1 \leq i \leq N} f_i(\mathbf{X}),
        \end{aligned}
    \]
    où
    \[
        f_i(\mathbf{X}) = \frac{1}{N} \sum_{1 \leq j \leq N} D(X_i, X_j) \Phi_{\mathcal{P}}(i, j)
    \]
\end{frame}

\begin{frame}

    \begin{itemize}
        \item Principe
            \begin{itemize}
            \item Utilisateur $i$ effectue seul le calcul de $f_i$
            \item Communication restreinte : voisinage dans $G$, taux de donn\'ees
            \end{itemize}
        \item Intérêt
            \begin{itemize}
                    \item Problème décomposé suivant chaque utilisateur
                    \item Nombreuses méthodes si problème convexe
                    \item Possibilité d'optimiser en ligne
            \end{itemize}
        \item Challenges
            \begin{itemize}
                \item $f_i$ non calculable par $i$
                \item Estimation de $f_i$
            \end{itemize}
    \end{itemize}

\end{frame}

\end{document}
