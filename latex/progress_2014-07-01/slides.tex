\documentclass[c]{beamer}

\usepackage[utf8]{inputenc}
\usepackage[french]{babel}

\usepackage{amsmath}
\usepackage{amsfonts}

\usepackage{graphicx}

\title{Estimation distribu\'ee d'une esp\'erance conditionnelle}
\author{Igor Colin}
\date{\today}

\setbeamertemplate{navigation symbols}{}

\AtBeginSection[]
{
    \begin{frame}<beamer>
        \tableofcontents[currentsection]
    \end{frame}
}

\begin{document}

\begin{frame}
  \maketitle
\end{frame}

\section{Rappels}

\begin{frame}
  \frametitle{Objectif et formulation}

  \begin{itemize}
    \item Objectif : regrouper les utilisateurs par centres d'int\'erêts communs
    \item Notations :
      \begin{itemize}
        \item $(X_i)_{1 \leq i \leq n}$ : caract\'eristiques des utilisateurs (musiques,
          historique des conversations, etc.)
        \item $D : (X, Y) \mapsto D(X, Y)$ : fonction de dissimilarit\'e entre deux
          vecteurs de caract\'eristiques
        \item $P$ : partition des utilisateurs
        \item $\Phi_P$ : fonction d'appartenance au même \emph{cluster}
      \end{itemize}
  \end{itemize}
\end{frame}

\begin{frame}
  \frametitle{Probl\`eme}

  \begin{itemize}
    \item Nouvel objectif : trouver la solution du probl\`eme
      \[
      \min_P w(P) = \frac{1}{n} \sum_{i =1}^n \frac{1}{n} \sum_{j=1}^{n} D(X_i, X_j) \Phi_P(X_i, X_j)
      \]
    \item Id\'ee : estimer $f : x \mapsto \mathbb{E}[D(x, X) \Phi_P(x, X)]$
    \item Contrainte : les $(X_i)_{1 \leq i \leq n}$ ne sont pas simultan\'ement accessibles
  \end{itemize}
\end{frame}

\section{Estimation de fonction}

\begin{frame}
  \frametitle{M\'ethode g\'en\'erale de regression}

  \begin{columns}
    \begin{column}{.6\textwidth}
      \begin{itemize}
        \item Notations :
        \begin{itemize}
          \item<1-> $f$ : fonction à estimer
          \item<2-> $\left\{ \left( x_i, f(x_i) \right) \right\}_{1 \leq i \leq n}$ : observations
          \item<3-> $\hat{f} : (x; \theta) \mapsto \hat{f}(x; \theta)$ : estimateur
          \item<4-> $\hat{R} : \theta \mapsto \hat{R}(\theta)$ : risque empirique
        \end{itemize}
      \end{itemize}
    \end{column}
    \begin{column}{.4\textwidth}
      \only<1>{
        \includegraphics[width=.9\textwidth]{./figures/regression_f.pdf}
      }
      \only<2>{
        \includegraphics[width=.9\textwidth]{./figures/regression_obs.pdf}
      }
      \only<3->{
        \includegraphics[width=.9\textwidth]{./figures/regression_several-thetas.pdf}
      }
    \end{column}
  \end{columns}
  \begin{itemize}
    \item<5-> Objectif : trouver $\theta^*$ solution de
      \[
        \min_{\theta \in \Theta} \hat{R}\left( \theta \right)
      \]
  \end{itemize}

\end{frame}

\begin{frame}
  \frametitle{Exemple}

  \begin{itemize}
    \item Exemple : estimation polynomiale
      \begin{itemize}
        \item $\hat{f} : (x; \theta) \mapsto \theta_0 + \theta_1 x + \theta_2 x^2$,
        \item $\hat{R}(\theta) = \sum_{i = 1}^n \left( \hat{f}(x_i) - f(x_i) \right)^2$
      \end{itemize}
    \item Qualit\'e d\'ependante du choix de $\hat{f}$
  \end{itemize}
  \begin{columns}
    \begin{column}{.5\textwidth}
      \begin{figure}
        \centering
        \includegraphics[width=.9\textwidth]{./figures/regression_good-choice.pdf}
        \caption{$\hat{f}$ adapt\'ee.}
      \end{figure}
    \end{column}
    \begin{column}{.5\textwidth}
      \begin{figure}
        \centering
        \includegraphics[width=.9\textwidth]{./figures/regression_poor-choice.pdf}
        \caption{$\hat{f}$ non adapt\'ee.}
      \end{figure}
    \end{column}
  \end{columns}
\end{frame}

\begin{frame}
  \frametitle{Application au problème initial}

  \begin{itemize}
    \item Fonction à estimer : $f : x \mapsto \mathbb{E}\left[ D(x,X) \Phi_P(x,X) \right]$
    \item Risque empirique : moindres carr\'es
    \item Estimateur à noyaux :
      \[
        \hat{f}(x; \mathbf{\theta}, \mathbf{w}) = \sum_{k = 1}^K w_k K(x - \theta_k)
      \]
      où $K$ est un noyau gaussien.
  \end{itemize}
\end{frame}

\end{document}
