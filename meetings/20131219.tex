\documentclass[a4paper]{article}

\usepackage[utf8]{inputenc}
\usepackage[french]{babel}

\usepackage[left=3cm, right=3cm]{geometry}

\title{Compte-rendu de la r\'eunion du 19 d\'ecembre 2013}
\author{}
\date{}

\begin{document}

\maketitle

\begin{description}
    \item[Pr\'esents]\hfill
        \begin{itemize}
            \item Thomas Bouton
            \item St\'ephan Cl\'emençon
            \item Igor Colin
            \item Joseph Salmon
        \end{itemize}
    \item[Ordre du jour]\hfill
        \begin{itemize}
            \item Apporter quelques pr\'ecisions suppl\'ementaires sur l'application
            \item Exposer les possibilit\'es de d\'eveloppement
        \end{itemize}
\end{description}

\section{Pr\'ecision sur l'application}
\label{sec:precision_application}

\begin{description}
    \item[Donn\'ees socio-d\'emographiques :] Les donn\'ees de ce type pouvant
        être r\'ecolt\'ees sont limit\'ees. L'application a accès à la carte SIM,
        au carnet d'adresse et au GPS. Le carnet d'adresse peut donner de
        bonnes indications sur le sexe des contacts et le GPS peut permettre
        de d\'eterminer la position du lieu de travail ou du domicile.
    \item[Recommandations :] Une vingtaine de cat\'egories existe \emph{a priori}.
        Concernant les publicit\'es, il existe un arbre de cat\'egories dont
        la profondeur permet d'être très sp\'ecifique. Il sera donc utile
        de mettre de la structure dans ces cat\'egories pour identifier
        les combinaisons de goûts les plus communes.
\end{description}

\section{Possibilit\'es de recherche}
\label{sec:possibilite_recherche}

\begin{description}
    \item[Simulation de graphes :] Ce champ est très ouvert ; il n'existe
        pas de m\'ethode permettant de g\'en\'erer rapidement des donn\'ees
        proches des donn\'ees r\'eelles. Un \'etat de l'art pr\'ecis de ce
        domaine doit être fait pour pouvoir reconstituer le graphe et ses
        propri\'et\'es.
    \item[D\'etection de communaut\'es :] Plusieurs approches sont possibles
        (it\'erative, spectrale). Il est n\'ecessaire, au moins dans la première
        phase de croissance de l'application, d'avoir une approche
        en ligne permettant de mettre à jour au fur et à mesure les
        communaut\'es.
    \item[Retour utilisateurs :] Lors du partage, les utilisateurs
        peuvent \'emettre un retour potentiellement riche en information.
        Ce retour doit être simple à effectuer pour l'utilisateur et
        devra \'egalement être mod\'elis\'e par la suite.
\end{description}

\section*{TODO}
\label{sec:TODO}

\begin{description}
    \item[StreamWide]\hfill
        \begin{itemize}
            \item Envoyer le fichier d'installation de l'application Android (iOS ?)
            \item Extraire les donn\'ees de l'application
        \end{itemize}
    \item[T\'el\'ecom]\hfill
        \begin{itemize}
            \item Mettre en place une m\'ethode pour simuler des graphes aux
                propri\'et\'es similaires au graphe d'utilisateurs existant.
            \item Impl\'ementer une maquette fonctionnant sur un graphe
                g\'en\'er\'e puis sur les donn\'ees r\'eelles.
        \end{itemize}
\end{description}

\end{document}
