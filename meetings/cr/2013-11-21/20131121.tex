%        File: 20131121.tex
%     Created: Thu Nov 21 11:00 AM 2013 C
% Last Change: Thu Nov 21 11:00 AM 2013 C
%
\documentclass[a4paper]{article}

\usepackage[utf8]{inputenc}
\usepackage[french]{babel}

\title{Compte-rendu de la r\'eunion du 21 novembre 2013}
\author{}
\date{}

\begin{document}

\maketitle

\begin{description}
    \item[Pr\'esents]\hfill
        \begin{itemize}
            \item Pascal B\'eglin
            \item Thomas Bouton
            \item St\'ephan Cl\'emençon
            \item Igor Colin
            \item Joseph Salmon
        \end{itemize}
    \item[Ordre du jour]\hfill
        \begin{itemize}
            \item Pr\'esentation des diff\'erentes techniques utilis\'ees pour
                les moteurs de recommandations
            \item Pr\'esentation de l'application et formalisation du problème
        \end{itemize}
\end{description}

\section{Fonctionnement de l'application}
Deux utilisateurs peuvent s'\'echanger des contenus de diverses natures :
texte (80\%), audio, vid\'eo, g\'eolocalisation ou fichiers.
Ces informations sont transf\'er\'ees entre les usagers via un serveur.
Les informations de type texte et g\'eolocalisation passant
par ce serveur sont sauvegard\'ees. Concernant les donn\'ees audio, vid\'eo et
les fichiers, seules les m\'etadata sont conserv\'ees.

L'application a \'egalement pour but d'être adapt\'ee à chacun. L'objectif est
que l'application puisse effectuer des recommandations de contenu (manag\'e ou
g\'en\'er\'e par les utilisateurs) adapt\'ees à un besoin local. L'utilisateur
pourra choisir de faire suivre ou non une recommandation lui \'etant faite. Il
pourra \'egalement s\'electionner le groupe d'utilisateurs vers qui il souhaite
faire suivre la recommandation.

\section{Donn\'ees disponibles et besoins pour l'application}
Les donn\'ees pouvant être exploit\'ees sont :
\begin{itemize}
    \item les informations sauvegard\'ees lors de leur passage sur le serveur
    \item les graphes utilisateurs
    \item les contenus à sugg\'erer, \'eventuellement accompagn\'es de
        descriptions et de cat\'egories
\end{itemize}
Le r\'esultat souhait\'e est la production de suggestions pertinentes en se
basant sur les donn\'ees disponibles. La mesure de la pertinence des
suggestions peut se baser sur :
\begin{itemize}
    \item le taux de clic (disponible en temps r\'eel)
    \item le partage du contenu propos\'e (disponible en temps r\'eel,
        cible du partage inconnue)
    \item notation des recommandations (à d\'eterminer)
    \item temps d'affichage (selon annonceur)
    \item achat du produit (selon annonceur)
\end{itemize}
Il est envisageable d'effectuer une partie des calculs directement sur le
t\'el\'ephone comme par exemple l'analyse s\'emantique des messages. La
capacit\'e de calculs sur les serveurs ne devrait pas être limitante pour
l'instant.

Il est n\'ecessaire d'avoir un outil capable de recommander du contenu
rapidement, quite à produire quelque chose de très simple dans un premier
temps.
Le modèle retenu pour le moteur de recommandations devra être capable de
s'adapter à la forte croissance potentielle du nombre d'utilisateurs.

\section*{TODO}
\label{sec:todo}
\begin{description}
    \item[StreamWide]\hfill
        \begin{itemize}
            \item Extraire les donn\'ees collect\'ees jusqu'à pr\'esent
            \item Envoyer le fichier d'installation de l'application Android
        \end{itemize}
    \item[T\'el\'ecom]\hfill
        \begin{itemize}
            \item Approfondir la recherche bibliographique
            \item Pr\'eparer une pr\'esentation technique plus sp\'ecifique par
                rapport aux m\'ethodes à employer
            \item Cr\'eer un r\'epertoire Dropbox pour regrouper les
                connaissances
        \end{itemize}
\end{description}

\end{document}


