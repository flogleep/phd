%        File: 20140330.tex
%     Created: Tue Apr 15 12:00 PM 2014 C
% Last Change: Tue Apr 15 12:00 PM 2014 C
%
\documentclass[a4paper]{article}

\usepackage[utf8]{inputenc}
\usepackage[french]{babel}

\title{Compte-rendu de la r\'eunion du 31 mars 2014}
\author{}
\date{}

\begin{document}

\maketitle

\begin{description}
    \item \textbf{Présents}
        \begin{itemize}
            \item Pascal Béglin
            \item Thomas Bouton
            \item Stéphan Clémençon
            \item Igor Colin
            \item Lilian Gaichies
            \item Joseph Salmon
        \end{itemize}
    \item \textbf{Ordre du jour}
        \begin{itemize}
            \item Présenter des techniques de g\'en\'eration de graphes
            \item Déterminer les donn\'ees du graphe exploitables
        \end{itemize}
\end{description}

\section{Données de l'application}

La récupération des données est en cours mais est rendue difficle par
différents changements dans le format des données.

\section{Tribus}
\label{sec:tribus}
L'application intègre un nouveau concept : la tribu. Les tribus sont un moyen
de regrouper les utilisateurs de manière plus souple et moins intrusive que les
cercles ou les liens d'amitié. Il n'est pas nécessaire d'avoir son nom réel
affiché pour les autres utilisateurs de la tribu. Il y a possibilité de faire
une tribu privée sur invitation ou invitation d'invitation ; il est également
envisageable de faire des groupes complètement privés, avec invitation par des
modérateurs uniquement.

Cette notion de tribu pourra être utilis\'ee pour effectuer des
recommandations ou des créations automatiques de tribus.

\section{D\'etection de communaut\'es}
Un objectif à court terme de l'application est de proposer automatiquement des
tribus aux utilisateurs. Pour cela, il serait int\'eressant d'utiliser des techniques
de d\'etection automatique de communaut\'es. Ces techniques devront cependant r\'epondre
à deux critères :
\begin{itemize}
    \item Elles devront permettre la superposition de communaut\'es : ainsi une tribu
        pourra être caract\'eris\'ee par l'intersection des communaut\'es la formant.
    \item Ces approches devront aussi int\'egrer les donn\'ees existantes sur les
        utilisateurs. En effet, l'objectif des tribus n'est pas de rassembler des
        personnes proches par leurs contacts en commun mais par leurs centres
        d'int\'erêts. De plus, l'utilisation de ces donn\'ees permettra de nommer
        explicitement chaque communaut\'e et donc les tribus.
\end{itemize}

Un objectif à plus long terme de l'application sera, une fois les tribus en place,
d'effectuer des recommandations adapt\'ees à un utilisateur. Les techniques de
partitionnement pr\'ecedemment \'evoqu\'ees pourrait alors se r\'eveler utiles pour \'etudier
les tribus pr\'esentes au sein d'un même \emph{cluster}.

Enfin, la question de la mesure de qualit\'e doit \'egalement se poser. Il reste à d\'eterminer
des critères permettant de quantifier la pertinence des tribus. Cela implique
\'egalement de d\'ecider les \'el\'ements repr\'esentant les connexions entre utilisateurs
(contact, occurence de messages, etc.).

\section{Distribution des donn\'ees et du calcul}
\label{sec:distribution_donnees_calcul}
Actuellement, certaines donn\'ees pr\'esentes sur le t\'el\'ephone sont accessibles par l'application
mais ne peuvent être transf\'er\'ees au serveur ; c'est le cas des SMS par exemple.
La capacit\'e de calcul des smartphones \'etant de manière g\'en\'erale très correct, la question
de la distribution des calculs se pose : il faut d\'efinir des m\'ethodes permettant
de r\'epartir le calcul sur le graphe tout en imposant des contraintes sur le transfert des
donn\'ees.

\section*{TODO}
\label{sec:todo}
\begin{description}
    \item \textbf{T\'el\'ecom}
    \begin{itemize}
        \item Faire le point sur la distribution multi-t\^aches pour l'optimisation
    \end{itemize}
    \item \textbf{Streamwide}
    \begin{itemize}
        \item R\'ecup\'erer les donn\'ees
        \item D\'eterminer les \'elements pertinents à consid\'erer pour les connexions
            entre utilisateurs
    \end{itemize}
\end{description}


\end{document}


