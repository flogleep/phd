%        File: 20140331.tex
%     Created: Fri Apr 04 12:00 PM 2014 C
% Last Change: Fri Apr 04 12:00 PM 2014 C
%
\documentclass[a4paper]{article}

\usepackage[utf8]{inputenc}
\usepackage[french]{babel}
\usepackage[left=3cm, right=3cm, top=4cm, bottom=4cm]{geometry}

\begin{document}

\section*{Community detection}
\label{sec:Community detection}
Une partie des données est sur le téléphone et ne peut être transférée telle
quelle au serveur (les SMS par exemples).
La détection de communauté est motivée par deux possibilités applicatives :
\begin{itemize}
    \item Le filtre de contact (auto-tribus)
    \item La proposition de contenu en fonction de la communauté détectée
\end{itemize}
Attention cependant, les méthodes présentées lors du meeting précédent étaient
des méthodes de \emph{clustering}: elles proposent une partition du graphe ce
qui implique que l'on ne peut pas obtenir de communautés se recouvrant les unes
les autres. Ce comportement pourrait pourtant être bénéfique pour indexé les
utilisateurs selon des thématiques concrètes.

[Problème: distribution par rapport aux données]

Un autre point abordé sur la détection de communauté est la notion de retour
des utilisateurs. En effet, la détection de communauté avec \emph{feedback} est
un domaine ouvrant de nouvelles considérations (processus de décision
markovien, apprentissage par renforcement). Il y aura un retour des
utilisateurs par rapport aux suggestions qui seront faîtes. Il aura trois
niveaux: négatif, neutre et positif.

Interrogation de Pascal Béglin: il n'y pas d'approche liée aux données dans ce
qui est discuté pour l'instant.

Pour Lilian, l'application est la combinaison de trois points de vue par
rapport aux messages:
\begin{itemize}
    \item Intimacy: idée de cercles plus ou moins intimes
    \item Nature du message/contenu
    \item Ce que propose une personne
\end{itemize}
Les trois approches existent dans les différents moteurs de recommandations
existants sur le marché mais ne sont jamais combinés tous ensemble.

De manière plus globale, il y a deux problèmes auxquels l'application devra
répondre :
\begin{itemize}
    \item Pousser des tribus intialement. Cela signifie donc être capable de
        taguer concrètement une tribu. Il faut donc également pour superposer
        les communautés ainsi trouvées.
    \item Une fois la période de croisière atteinte, regarder les tribus
        existantes dans un cluster donner afin d'adapter les suggestions à
        l'utilisateur.
\end{itemize}

Autrement dit, il faut organiser un ??? (Pinterest, \emph{Timeline}) sauf que
l'utilisateur peut devenir contact avec des personnes en particulier des tribus
pour les connaître en particulier.

Une autre question va également se situer au niveau de la mesure de qualité. Il
faudra être capable de comparer les tribus. Les connexions entre les
utilisateurs qui seront représentées dans le graphe doivent être également
déterminées. En effet, une connexion représente un lien (plus ou moins fort si
l'on y associe un poids) mais ce lien peut signifier plusieurs choses --
intensité, occurrence de messages sur une fenêtre de temps plus ou moins longue
-- qui sont à choisir par StreamWide selon la valeur qu'ils accordent aux
différentes relations entre les utilisateurs.

Du côté Télécom, il faut faire le point sur la distribution multit\^ache pour
l'optimisation.

\section*{Tribus}
\label{sec:Tribus}
Les tribus sont un moyen de regrouper les utilisateurs de manière plus souple
et moins nécessairement intrusive que les cercles ou les liens d'amitié. Il
n'est pas nécessaire d'avoir son nom réel affiché pour les autres utilisateurs
de la tribu. Cela également pour le pseudonyme. Il y a possibilité de faire une
tribu privée sur invitation ou invitation d'invitation ; il est également
envisageable de faire des groupes complètement privées, avec invitation par des
modérateurs uniquement.

\end{document}


