\documentclass[c]{beamer}

\usetheme{default}
\usefonttheme{structurebold}

\usepackage[utf8]{inputenc}
\usepackage{lmodern}

\usepackage[]{amsmath}
\usepackage{amsfonts}

\author{Igor Colin}
\date{\today}

\begin{document}

\begin{frame}
    \frametitle{Principe de l'application}

    \begin{itemize}
        \item Partage entre utilisateurs
            \begin{itemize}
                \item texte
                \item image
                \item vidéo
                \item géolocalisation
                \item fichier
            \end{itemize}
        \item Recommandations d'événements
            \begin{itemize}
                \item personnalisées
                \item possibilité de faire suivre à d'autres utilisateurs
            \end{itemize}
        \item Objectif
            \begin{itemize}
                \item modéliser les échanges d'informations
                \item établir profils et liens entre utilisateurs
                \item faire des recommandations personnalisées
            \end{itemize}
    \end{itemize}
\end{frame}

\begin{frame}
    \frametitle{Graphes}

    \begin{itemize}
        \item Graphe $G = (V,E)$
        \item N\oe{}uds $V = \{v_1,\ldots,v_n\}$
        \item Arrêtes $E = \{e_1,\ldots,e_m\}$
            \begin{itemize}
                \item $e = (u,v) \in V \times V$
                \item $e = (u,v,l) \in V \times V \times L$
                \item si $L \subset \mathbb{R}$, matrice d'adjacence
            \end{itemize}
    \end{itemize}
\end{frame}

\begin{frame}
    \frametitle{Statistiques}

    \begin{itemize}
        \item Degré :
            \begin{itemize}
                \item entrant, sortant
                \item moyenne, minimum, maximum
                \item distribution (loi puissance)
            \end{itemize}
        \item Diamètre
            \begin{itemize}
                \item réel
                \item efficace
            \end{itemize}
        \item Condutance / expansion
        \item Modularité
        \item Matrice laplacienne
    \end{itemize}
\end{frame}

\begin{frame}
    \frametitle{Génération de graphes}

    \begin{itemize}
        \item Objectif : pouvoir générer des graphes aux propriétés similaires
        \item Méthodes
            \begin{itemize}
                \item Attachement préférentiel
                \item \textit{Forest Fire}, Kronecker (Leskovec 2005, 2007)
            \end{itemize}
    \end{itemize}
\end{frame}

\begin{frame}
    \frametitle{Détection de communautés}

    \begin{itemize}
        \item Un graphe peu contenir des sous-graphes très connectés
        \item 2 utilisateurs d'un même sous-graphe peuvent avoir un profil
            similaire
    \end{itemize}
\end{frame}

\begin{frame}
    \frametitle{Propagation de l'information}

    \begin{itemize}
        \item Détection de communauté : structure du graphe
        \item Modéliser la propagation de l'information
            \begin{itemize}
                \item Théorie de la survie
                \item Influences dans un réseau (Kempe et al 2003)
            \end{itemize}
    \end{itemize}
\end{frame}

\begin{frame}
    \frametitle{Statique vs dynamique}

    \begin{itemize}
        \item Beaucoup de modèles/méthodes basées sur des graphes statiques
        \item En réalité, $G(t) = (V(t), E(t))$
        \item Nécessité de méthodes adaptées (online)
    \end{itemize}
\end{frame}

\end{document}
