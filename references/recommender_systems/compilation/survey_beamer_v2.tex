\documentclass[c]{beamer}

\usetheme{default}
\usefonttheme{structurebold}

\usepackage[utf8]{inputenc}
\usepackage{lmodern}

\usepackage[]{amsmath}
\usepackage{amsfonts}

\author{Igor Colin}
\date{\today}

\begin{document}

\begin{frame}
    \frametitle{Moteur de recommandations}

    \begin{itemize}
        \item Id\'ee : effectuer des suggestions pertinentes pour un
            utilisateur
        \item Donn\'ees disponibles :
            \begin{itemize}
                \item Historiques des utilisateurs
                \item Informations sur les suggestions disponibles
                \item Liens entre les utilisateurs
            \end{itemize}
        \item Challenges :
            \begin{itemize}
                \item Peu de donn\'ees sur chaque utilisateur
                \item Beaucoup de suggestions diff\'erentes possibles
            \end{itemize}
        \item Exemples : Amazon, Netflix, publicit\'e cibl\'ee
            \begin{center}
            \includegraphics[width=.8\textwidth]{/home/igorcolin/Downloads/amazon.png}
        \end{center}
        %Image Amazon
    \end{itemize}
\end{frame}

\begin{frame}
    \frametitle{Compl\'etion de matrice}

    \begin{itemize}
        \item Pour chaque contenu consomm\'e, un utilisateur \'emet un avis
            \begin{itemize}
            \item Explicite : note sur 5, j'aime
            \item Implicite : dur\'ee de consultation d'une page, d'une vid\'eo, partage de lien
        \end{itemize}
        \item<2-> Pr\'ediction des avis qu'\'emettrait un utilisateur pour les
            contenus non consomm\'es
        \item<3-> Suggestion des contenus aux avis pr\'edits les plus favorables et
            mesure de la performance
    \end{itemize}
    \[
        \begin{array}{|r|c|c|c|c|}
            \hline
            & C_1 & C_2 & C_3 & C_4 \\
            \hline
            U_1 & 4 & 5 & \only<1>{-} \only<2->{\textcolor{red}{3}}  & \only<1>{-} \only<2>{\textcolor{red}{5}} \only<3>{\fcolorbox{blue}{white}{\textcolor{red}{5}}}\\
            \hline
            U_2 & 3 & \only<1>{-} \only<2>{\textcolor{red}{3}} \only<3>{\fcolorbox{blue}{white}{\textcolor{red}{3}}}& 3 & \only<1>{-} \only<2->{\textcolor{red}{1}} \\
            \hline
        U_3 & \only<1>{-} \only<2>{\textcolor{red}{3}} \only<3>{\fcolorbox{blue}{white}{\textcolor{red}{3}}}& \only<1>{-} \only<2->{\textcolor{red}{1}} & \only<1>{-} \only<2->{\textcolor{red}{2}} & 1 \\
            \hline
        \end{array}
    \]
\end{frame}

\begin{frame}
    \frametitle{M\'ethodes bas\'ees sur le contenu}

    \begin{itemize}
        \item Principe : sugg\'erer des objets proches de ceux appr\'eci\'es
            dans le pass\'e
        \item<2-> M\'ethode :
            \begin{itemize}
                \item Caract\'eriser les contenus pour d\'efinir une mesure de
                    similarit\'e
                    \begin{itemize}
                        \item $s(\text{Harry Potter 1}, \text{Harry Potter 2}) = 0.9$
                        \item $s(\text{Harry Potter 1}, \text{Annie Hall}) = 0.1$
                    \end{itemize}
                \item Utiliser l'historique des contenus de
                l'utilisateur pour pr\'edire les avis
            \end{itemize}
        \item<3-> Avantages :
            \begin{itemize}
                \item Peu coûteux
                \item Nouveau contenu peut être recommand\'e
            \end{itemize}
        \item<4-> Inconv\'enients :
            \begin{itemize}
                \item Quelle similarit\'e ?
                \item N'utilise que les donn\'ees de l'utilisateur courant
                \item Impossible d'effectuer une recommandation pour un
                    nouvel utilisateur
            \end{itemize}
    \end{itemize}
\end{frame}

\begin{frame}
    \frametitle{M\'ethodes collaboratives historiques}

    \begin{itemize}
        \item Principe : sugg\'erer des contenus appr\'eci\'es par les
            utilisateurs aux goûts similaires
        \item<2-> M\'ethode :
            \begin{itemize}
                \item D\'eterminer les similarit\'es entre utilisateurs
                    à partir des avis sur des contenus communs
                \item Effectuer une pr\'ediction gr\^ace aux avis d\'ejà \'emis
                    par des utilisateurs similaires
            \end{itemize}
        \item<3-> Avantages :
            \begin{itemize}
                \item Utilise l'ensemble des donn\'ees à disposition
                \item Ne n\'ecessite pas de compr\'ehension des contenus
            \end{itemize}
        \item<4-> Inconv\'enients :
            \begin{itemize}
                \item Impossible de recommander un nouveau contenu
                \item Impossible d'effectuer une recommandation pour un
                    nouvel utilisateur
            \end{itemize}
    \end{itemize}
\end{frame}

\begin{frame}
    \frametitle{M\'ethodes collaboratives avec modèle}

    \begin{itemize}
        \item Principe : apprendre un modèle pour pr\'edire les avis
        \item<2-> M\'ethode :
            \begin{itemize}
                \item D\'efinir un modèle pr\'edictif
                    \begin{itemize}
                        \item ex : $avis(u,c) = m + f(u) + g(c)$
                    \end{itemize}
                \item Utiliser l'ensemble des donn\'ees pour apprendre
                    les paramètres du modèle et pr\'edire les avis manquants
            \end{itemize}
        \item<3-> Avantages :
            \begin{itemize}
                \item Utilise l'ensemble des donn\'ees à disposition
                \item Bonne qualit\'e de pr\'ediction lorsque modèle adapt\'e
            \end{itemize}
        \item<4-> Inconv\'enients :
            \begin{itemize}
                \item Souvent coûteux
                \item Impossible de recommander un nouveau contenu
                \item Impossible d'effectuer une recommandation pour un
                    nouvel utilisateur
            \end{itemize}
    \end{itemize}
\end{frame}

\begin{frame}
    \frametitle{Compl\'etion de matrice}

    \begin{itemize}
        \item M\'ethodes vari\'ees
        \item Adapt\'ees à un grand nombre de problèmes
        \item Plusieurs inconv\'enients communs :
            \begin{itemize}
                \item Pas de recommandation pour un nouvel utilisateur
                \item Ne propose que du contenu similaire
            \end{itemize}
        \item Regarder les approches par bandits (exploration/exploitation)
    \end{itemize}
\end{frame}

\end{document}
