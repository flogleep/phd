%        File: dcg.tex
%     Created: Fri Oct 25 11:00 AM 2013 C
% Last Change: Fri Oct 25 11:00 AM 2013 C
%
\documentclass[a4paper]{article}

\usepackage[utf8]{inputenc}

\usepackage{amsmath}
\usepackage{amsfonts}

\begin{document}

\section{Discounted cumulative gain}

Discounted cumulative gain is a measure of the quality of a search engine, in
information retrieval.

\subsection{Cumulative gain}
Let $\left\{ X_1, \ldots, X_N \right\}$ be the first $N$ results of a given
query and let $rel(\cdot)$ be the associated relevance function. The cumulative
gain at rank $p \le N$ is defined by:
\[
  \mathrm{CG}_p = \sum_{i = 1}^{p} rel(X_i).
\]
It is simply the cumulated relevance of the first $p$ results.

\subsection{Discounted cumulative gain}
Cumulative gain does not take into account the fact that a relevant result
should be penalized if it appears at a large rank. The discounted cumulative
gain is based on the cumulative gain definition but the relevance is penalized
logarithmically proportional to the rank of the result. There are two
definitions of the discounted cumulative gain. Even if they are quite similar,
they are not equivalent. The first definition is given by:
\[
  \mathrm{DGC}_p = rel(X_1) + \sum_{i = 2}^{p} \frac{rel(X_i)}{\log_2(i)}.
\]
The second definition is given by:
\[
  \mathrm{DGC}_p = \sum_{i = 1}^{p} \frac{2^{rel(X_i)} - 1}{\log_2(i + 1)}.
\]
%TODO: check theoretical explanation about log_2 on denominator

\subsection{Normalized discounted cumulative gain}
DCG can be useful to characterize a search engine's performance in response to
one query. However, it does not allow for consistent comparison for different
queries: the length of the results list can vary and this is not taken into
account by DCG alone. Let $\mathrm{IDCG}_p$ be defined for $p \le N$ as follow:
\[
  \mathrm{IDCG}_p = \max_{\sigma \in \mathfrak{S}} 
  DCG_p\left( \left( X_{\sigma(1)}, \ldots, X_{\sigma(p)} \right) \right).
\]
This represents the best DCG possible considering $p$ results from the full
list. We then define the normalized discounted cumulative gain by:
\[
  nDCG_p = \frac{DCG_p}{IDCG_p}.
\]
The nDCG represents the quality of the $p$ firsts results, compared to the $p$
best results from the full set.

Normalized discounted cumulative gain can be very useful when addressing a
ranking problem for a recommender. The problem is then to sort recommandations
and to show the top-$N$ ones, where $N$ is very small compared to the total
number of possible recommendations. Hence, it is important that top 
recommendations are more emphasized than other ones.

\section{Mean reciprocal rank}
The mean reciprocal rank is a statistic used for evaluating a process producing
a list of possible answers to a given query. The smallest the rank of the
correct answer, the higher the MRR. Let $\mathcal{Q}$ be a set of query and let
$\left( r_q \right)_{q \in \mathcal{Q}}$ be the corresponding ranks of the
correct answers. The MRR associated to $\mathcal{Q}$ is defined by:
\[
  \mathrm{MRR}(\mathcal{Q}) = \frac{1}{|\mathcal{Q}|}
  \sum_{q \in \mathcal{Q}} \frac{1}{r_q}.
\]

\end{document}
