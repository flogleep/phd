\documentclass[11pt,sans]{beamer}

\usepackage[utf8]{inputenc}

\usepackage[]{amsmath}
\usepackage{amsfonts}

\begin{document}
\begin{frame}
        \frametitle{StreamWide}

        \begin{itemize}
            \item Applications de partage entre utilisateurs
                \begin{itemize}
                    \item Chat
                    \item Photos
                    \item Vid\'eos ?
                    \item Musique ?
                \end{itemize}
            \item Recommandations portant sur :
                \begin{itemize}
                    \item Lieux
                    \item \'Ev\'enements
                    \item News
                    \item Musique ?
                \end{itemize}
            \item Recommandations bas\'ees sur :
                \begin{itemize}
                    \item Donn\'ees locales/globales ?
                    \item Informations partag\'ees via applications
                    \item G\'eolocalisation
                    \item Retour utilisateur (buzz, proposition, etc.)
                \end{itemize}
        \end{itemize}
    \end{frame}

    \begin{frame}
        \frametitle{M\'ethodes de recommandations}
        \framesubtitle{Bas\'ee sur le contenu}

        \begin{itemize}
            \item Principe
                \begin{itemize}
                    \item Repr\'esente un contenu (item) dans un espace adapt\'e
                        pr\'ed\'efini (ex : TF-IDF pour NLP)
                    \item Regarde historique utilisateur courant
                    \item Propose contenu \emph{proche} de l'historique
                \end{itemize}
            \item Avantages
                \begin{itemize}
                    \item Utilise uniquement les donn\'ees locales
                    \item Simple à impl\'ementer
                    \item Pas de problème de nouvel objet
                \end{itemize}
            \item Inconv\'enients
                \begin{itemize}
                    \item Si donn\'ees non denses (peu d'objets
                        dans l'historique par rapport au nombre total
                        d'objets), recommandations peu pertinentes
                    \item N'utilise pas les donn\'ees d'autres utilisateurs
                    \item Problème du nouvel utilisateur
                \end{itemize}
        \end{itemize}
    \end{frame}

    \begin{frame}
        \frametitle{M\'ethodes de recommandations}
        \framesubtitle{Collaborative bas\'ee sur la m\'emoire}

        \begin{itemize}
            \item Principe
                \begin{itemize}
                    \item Regarde historique utilisateur courant
                    \item Cherche utilisateurs avec historique similaire
                    \item Propose contenu des utilisateurs \emph{proche}
                \end{itemize}
            \item Avantages
                \begin{itemize}
                    \item S'appuie sur d'autres utilisateurs, donc plus
                        de donn\'ees
                    \item Simple à impl\'ementer
                    \item Pas besoin de mod\'eliser les contenus
                \end{itemize}
            \item Inconv\'enients
                \begin{itemize}
                    \item Ne peut pas s'appuyer uniquement sur les donn\'ees
                        locales
                    \item Problème du nouveau contenu
                    \item Problème du nouvel utilisateur
                \end{itemize}
        \end{itemize}
    \end{frame}

    \begin{frame}
        \frametitle{M\'ethodes de recommandations}
        \framesubtitle{Collaborative bas\'ee sur un modèle}

        \begin{itemize}
            \item Principe
                \begin{itemize}
                    \item Mod\'elise le lien entre utilisateurs et objets
                    \item Regarde l'historique complet pour apprendre les
                        paramètres du modèle
                    \item Pr\'edit les pr\'ef\'erences d'un utilisateur à
                        partir du modèle ainsi appris
                \end{itemize}
            \item Avantages
                \begin{itemize}
                    \item Utilisation de toutes les donn\'ees disponibles
                    \item Peut donner une intuition sur les critères de
                        recommandations
                    \item Bons r\'esultats de pr\'ediction
                \end{itemize}
            \item Inconv\'enients
                \begin{itemize}
                    \item Choix du modèle crucial pour bon \'equilibre
                        qualit\'e/temps d'ex\'ecution
                    \item Problème du nouveau contenu
                    \item Problème du nouvel utilisateur
                \end{itemize}
        \end{itemize}
    \end{frame}

    %\begin{frame}
        %\frametitle{Techniques de recommandations}

        %\begin{itemize}
            %\item Content-based:
                %\begin{itemize}
                    %\item TODO
                %\end{itemize}
            %\item Collaborative filtering:
                %\begin{itemize}
                    %\item Memory-based:
                        %\begin{itemize}
                            %\item Nearest neighbors
                            %\item Top-N recommendations
                        %\end{itemize}
                    %\item Model-based:
                        %\begin{itemize}
                            %\item Bayesian beliefs net
                            %\item Latent semantic
                            %\item SVD++
                        %\end{itemize}
                %\end{itemize}
            %\item Reinforcement learning??
        %\end{itemize}
    %\end{frame}

    %\begin{frame}
        %\frametitle{Content-based vs CF}

        %M\'ethodes bas\'ees sur le contenu
        %\begin{itemize}
            %\item Avantages
                %\begin{itemize}
                    %\item Pas de problème de nouvel objet
                    %\item Simple à impl\'ementer
                    %\item Ne n\'ecessite pas de donn\'ees d'autres utilisateurs
                %\end{itemize}
            %\item Inconv\'enients
                %\begin{itemize}
                    %\item N\'ecessite une compr\'ehension \emph{a priori} de
                        %l'objet
                    %\item Utilise peu de donn\'ees : repose beaucoup sur la
                        %repr\'esentation de l'objet
                %\end{itemize}
        %\end{itemize}

        %Filtrage collaboratif
        %\begin{itemize}
            %\item Avantages
                %\begin{itemize}
                    %\item Utilise les informations des autres utilisateurs
                    %\item Ne n\'ecessite pas de connaissances \emph{a priori}
                        %des objets
                %\end{itemize}
            %\item Inconv\'enients
                %\begin{itemize}
                    %\item Problème du nouvel objet
                %\end{itemize}
        %\end{itemize}
    %\end{frame}

    %\begin{frame}[<+->]
        %\frametitle{Memory-based CF vs model-based CF}

        %Filtrage collaboratif bas\'e sur le pass\'e
        %\begin{itemize}
            %\item M
        %\end{itemize}
    %\end{frame}

    %\begin{frame}
        %\frametitle{Collaborative filtering}
        %\framesubtitle{Memory-based}

        %\begin{itemize}
            %\item Advantages:
                %\begin{itemize}
                    %\item Easy to implement
                    %\item New data can be added incrementally
                    %\item Do not need to ``understand'' the nature of the items
                %\end{itemize}
            %\item Shortcomings
                %\begin{itemize}
                    %\item Dependent on human ratings (in the case of score
                        %prediction)
                    %\item New user/item problem
                    %\item The sparser the worse
                %\end{itemize}
        %\end{itemize}
    %\end{frame}

    %\begin{frame}
        %\frametitle{Collaborative filtering}
        %\framesubtitle{Model-based}

        %\begin{itemize}
            %\item Advantages:
                %\begin{itemize}
                    %\item Improved prediction performance
                    %\item Better scalability
                    %\item Intuition on recommendations
                %\end{itemize}
            %\item Shortcomings
                %\begin{itemize}
                    %\item Trade-off between prediction performance and
                        %scalability
                    %\item Lost of useful information for dimensionality
                        %reduction techniques
                %\end{itemize}
        %\end{itemize}
    %\end{frame}
\end{document}
